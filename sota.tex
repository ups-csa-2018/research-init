\documentclass[a4paper, twocolumn]{article}

\usepackage{biblatex}
\usepackage{censor}
\usepackage{lipsum}
\usepackage{tagging}
\usepackage[top=2cm, bottom=3cm, left=2cm, right=2cm]{geometry}
\setlength{\columnsep}{1cm}

\bibliography{sota}

\usetag{draft}
\tagged{draft}{
    \usepackage{draftwatermark}
    \SetWatermarkLightness{0.9}
    \SetWatermarkScale{4}
}

\begin{document}

\tagged{redacted}{}
\untagged{redacted}{\StopCensoring}

\title{Tweet summarization methods' state of the art \\
    \large Université de Toulouse 3}
\author{
    \censor{Jules Lamur},
    \censor{Oscar Che},
    \censor{Quentin Mestre} and
    \censor{Jeremy Duran}}
\date{}

\maketitle

\begin{abstract}

\end{abstract}

\section{Introduction}

Social media has become a big part of our society's daily lifestyle, on average
a person has 7.6 accounts and spends 2 hours a day on social medias. In the top
5 is Twitter along with Facebook, youtube, Instagram and snapchat. Since it’s
launch in 2006 Twitter has accumulated 1.3 billion accounts with 126 million
daily active users.

The microblogging website limits its users to 280 characters per posts (or as
twitter calls them tweets), but that does not stop its users as 500 million
tweets are generated each day. In fact, twitter users say and react to whatever
they like, may it be on the new Avengers trailer, videos, news or events,
making it a great resource for researchers.

With the rapid growth of tweets, it is sometimes hard for users to grasp
essential information, a solution is twitter summarization. It aims to give a
concise summary from a huge amount of tweets in a given topic. Applications
could be helping users to understand topics, help agencies monitor events,
crisis progress or even disaster relief moreover due to the huge amount of data
and the rapidness of the information, it can also be interesting to look into
Realtime summarizations.

Document summarization has been around for several years, and despite having
spawned multiple research papers, it still makes of a worthy opponent due to
the nature of tweets and how researchers approach their noisy-emoticon-short
nature.

\section{Topic's characteristics}

Twitter isn’t only a social network which provides usual “social” features such
as chatting , having a profile , write freely about anything, sharing medias...
it’s also a really popular microblogging platform. Nowadays, it’s really easy
for one’s  to post instantly and frequently about their own interests, and due
to this easiness , the amount of posts in microblogs became very huge compared
with that of the traditional blogs.That huge amount of data represents a huge
pool of information , however since each tweet has a short format , the
relevant information density is low and for this reason it’s hard to catch
trends and public opinions on a topic.There is also the follow feature that
allows one user to follow friends, or celebrities or a person that represents
huge interest from the user. Last really twitter-specific feature is the
‘retweet’ that allows the user to take someone’s tweet and share it with his
followers in order to spread that tweet, and the user can add anything he wants
to legend that tweet. All those features create strong topic-related connexions
between microblogs that will increase again the information pool. That’s why
summarization is a rich research field and interesting in order to not be
flooded by this huge amount of information.

\section{Summarization methods}

We distinguished two approaches in performing tweeter summarization :
Extractive and abstractive.We’ll discuss about their main features ,
differences ,similarities and methods they used in order to produce a twitter
summary. However their relevance is function of the user’s  appreciation.

\subsection{Extractive methods}

Extractive summarization is the process of selecting sentences, words , or even
tweets  ( in this case )  as a whole   from the given data set and to put them
together. The summary can take many forms like : Top tweets related on a topic
, Most relevant tweet , group of tweets  , group of keywords...The main feature
of an extractive summary is that the content of the summary is truly authentic.

\subsubsection{Graph based approach}

One way to extract a good and readable summary is to establish a graph that
represents all the sentences of selected tweets, in which all words is a node
and are related to other words. All these nodes will be weighted by their
number of occurrences in the set of sentences. (Sharifi et al. in 2015)[1]
describes a method based on a graph that extract a summary from a set of
tweets/sentences. It consists in first get all  the results (tweets) of a
research on a specific topic using Twitter’s API (it returns 1500 results
maximum), then it execute the “phase reinforcement algorithm”. This algorithm
build a graph using as root node the most occurrent word (or set of words). The
graph is organised form left to right, and in the middle is the root node, the
leafs on the left hand side are the beginning of the sentences, and the leafs
on the right hand side are the ends of the sentences. A word w of a sentence s
is linked in the graph with its previous neighbor in s, and its next neighbor
in s. All nodes are weighted according to their frequency of occurrence
respective of their word ordering from the root. Now the graph is build, the
algorithm has to find the most weighted path and then build a new graph using
the same way putting the phrase resulting to this path as the root of the new
graph. It will be repeated until the graph is not empty. Then the summary is
generated.

\subsubsection{Clustering approach}

There are numerous techniques that use clustering approaches in order to
produce an extractive summary. Clustering is the task of grouping a set of
objects ( here it will be a set of tweets ) that are more similar between them,
then it will  form a cluster.
The paper by Kim ,Tae-Yeon in 2014[2] introduce a clustering approach to
produce an extractive summary. Their method consist in:  first they form a
keyword graph thanks to a keywords analysis (Term frequency-inverse document
frequency also known as TF-IDF) that will produce a graph of keywords , then
they use the K-clique Clustering method in order to group the tweets.( A
k-clique is a sub-graph which has k nodes and the nodes are fully connected to
the others) .Tweets that contain all the words of that clique will have high
probability to be similar to each others and then, they’re more likely to form
a cluster.This method produce as a result , a cluster of tweets depending of
the keyword that the user searched for . Another approach that is worth
notifying for , is from the work of Xintian yang in 2012, that uses a
clustering approach based on a time window , their algorithm is called
(SPUR)[3] . In this method , tweets will be clustered by 1 hour time interval,
(they consider that tweets that were posted within 1 hour will be more similar
between them than tweets appearing 2 hours later) then  they extract the most
used sentence from that cluster and evaluate and rank each tweet by their
utility and values, at the end, they include the most relevant tweets in their
summary.


\subsubsection{Statistical approach}

\subsection{Abstractive methods}

We can define Abstractive summaries as “ Someone is making for you a summary
after being on twitter in your place “ in fact , Abstractive summaries will
generate sentences that may not appear in the input data, the goal here is to
generate human-understandable sentences with high information density so the
user has a global idea of what’s happening while decreasing his search efforts.
We must notify that those methods are fewer than the extractives one.

\subsubsection{Graph based approach}

The summarizations methods that will be presented here will not much differ
from the one presented in the Extractive section , but it’s interesting to see
how their summaries differ. We’ll have a look on Andrei Olariu ‘ s work in
2014[X], and it’s Twitter Online word Graph Summarizer ( TOWGS ), this approach
is interesting because it doesn’t save any of the tweets, it also skips any
clustering step, they instead build word graphs from trigrams ( n-grams of
words )  , each trigrams will represent a node and their corresponding edges
will be weighted based on the frequencies of that trigram related to the
tweets. But the more important part is that this method is Online , in the
previous techniques we presented the word graphs were discarded after
generating the summary, that’s not the case here: for the online summary , the
graph is constantly updated with tweets and their initial word graph “forget”
the old data via a time decaying window. Then in order to generate an
abstractive summary this technique is looking for the highest scoring path by
using a greedy search strategy  in the graph, this path connects the special
words which marks the beginning and the end of the sentence and propose an
updated summary sentence as a result. TOWGS method is a real contender for the
most advanced abstractive real time method on tweets summarization.

\subsubsection{Others}

A paper from Nilambari Maruti Dhanve from the Solapur University in India, uses
Speech Acts in his paper Twitter Trending Topic Summarization Using Speech Act
[X]  for his abstractive summarization. A Speech Acts is something expressed by
an individual that not only presents information, but perform an action as
well. For example "Hi, Eric. How are things going?" is performing the action of
Greeting, and "Could you pass me the mashed potatoes, please?" corresponds to
request.  They use a data set to extract phrases (n-grams) and assign them a
part of speech either being a verb, noun,comment or question. They then
implement a ranking to the words and select the most salient ones for the
summary. To construct the abstractive summary the apply a template of :
for”<topic word>”,people<verb frame>”ngrame”{,(and)<verb frame>”<ngrams>”} to
the ngrams.  The “ngrams” are the salient words/phrases extracted for the major
speech act types. A “verb frame” is a verb or verb phrase specific to a
particular speech act type.The  algorithm  favors longer ngrams so that the
generated summary contains informative and less ambiguous phrases.

\section{Evaluation methods}

\section{Conclusion}

\printbibliography

\end{document}
